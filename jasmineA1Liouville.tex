\subsection{by Jasmine}

For a 3-dimensional system of $N$ particles described by a Hamiltonian $H(\mathbf{q}, \mathbf{p})$, we can define a microstate by the positions and momenta of each of the $N$ particles in the system. This microstate can be represented in the 6-dimensional phase space by a representative point $(q_1, \ldots, q_{3N}, p_1, \ldots, p_{3N} )$. For a macroscopic system, we can consider an ensemble in phase space - a collection of all the microstates corresponding to the system. We think of the ensemble as continuous, with a density governed by a density distribution function $\rho (\mathbf{q}, \mathbf{p}, t) $. Liouville's theorem states that this distribution function is constant along the trajectories of the system, i.e. that the density of the ensemble is constant with time: \[ \frac{d\rho}{dt} = \frac{\partial\rho}{\partial t} + [\rho, H]=0 \]
where $[\rho, H]$ is the Poisson bracket of the distribution function $\rho$ the Hamiltonian $H$ governing the system. Liouville's theorem lets us divide up the continuous phase space and treat it as filled with discrete objects (microstates).