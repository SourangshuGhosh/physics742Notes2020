\section{Ensembles}

For a given macrostate $(N, V, E)$, a statistical system, at any time t, may be in any one of an extremely large number of distinct microstates. As time passes, the system transitions between different microstates, all of which correspond to the same macrostate, with the result that, over a long enough time period the behaviour of the systems is ``averaged'' over the collection of microstates which have been visited by the system.
A useful approach is to consider, at a single instant of time, a large (infinite) number of copies of the same system, existing in all possible microstates that satisfy the macroscopic conditions. Then, we can expect the average behaviour of any system for this collection (or ensemble) to be identical with the time-averaged behaviour of the system. This forms the basis of the so–called ensemble theory.

\subsection{The micro-canonical ensemble}
In the previous section, we considered an isolated system where we could keep track of the dynamics of every particle and use that to calculate the values of extensive, macroscopic properties of the system. An important aspect of this was the conservation of energy for the system. We refer to such systems as a \emph{micro-canonical ensemble} --- the system is isolated with no heat flux and no change in the number of particles, hence the internal energy of the system is constant and it is described entirely by the Hamiltonian dynamics.

Since $H=E=\mbox{const.}$ for a micro-canonical ensemble, we can associate the ensemble with a hypersurface $H=E$ that corresponds to a the macrostate $(N,V,E)$. However, in order to allow for small fluctuations that correspond to what is essentially the same state, (we require any observables to be smoothly varying functions of the phase space) it is often desirable to replace the hyper-surface with the hyper-shell $E-\Delta/2 \leq H \leq E+\Delta/2$. The \emph{density function} $\rho(p,q,t)$ characterises the distribution of points in phase space for a system in some state. With three d.o.f. and $N$ particles, we therefore have $\rho(p,q,t)\mathrm{d}^{3N}p\mathrm{d}^{3N}q$ as the number of representative points in phase space in a volume element $\mathrm{d}^{3N}p\mathrm{d}^{3N}q$, centred at the point $(p,q)$ at time $t$. From Liouvilles theorem, we have the $\frac{\partial \rho}{\partial t}=0$. 

When the internal energy of the system lies in $E-\Delta/2 \leq H \leq E+\Delta/2$ we have


\[ 
	\rho(p,q) = 
\begin{cases} 
      \mbox{const.} & \mbox{ if } E-\Delta/2 \leq H \leq E+\Delta/2, \\
      0 & \mbox{otherwise.} 
   \end{cases}
\]

For the hyper-shell $E-\Delta/2\leq H \leq E+\Delta/2$ the enclosed volume is
\[
\omega(E) = \int_{E-\Delta/2\leq H \leq E+\Delta/2} \mathrm{d}^{3N}p\mathrm{d}^{3N}q.
\]
We'd like to establish a numerical correspondence between $\Gamma$ and $\omega$ in order to account for the multiple microstates that can correspond to a single macrostate of the system. To do so, we introduce the concept of a \emph{fundamental volume} $\omega_0$ which we regard as being the volume equivalent to one microstate. Then $\Gamma = \omega/\omega_0$ and hence
\[
S =k \ln|\Gamma| = k\ln(\omega/\omega_0).
\]


Now, let's consider how we might work with ensembles if we have a composite system made up of two subsystems. The simplest case is two (isolated) systems in contact, without any exchange of energy. If the state of system 1 corresponds to the region of phase space $\Gamma^1$ and similarly, the state of system 2 to $\Gamma^2$ then the state of the composite system $1\cup 2$ corresponds to the phase space regions given by the Cartesian product $\Gamma^{1\cup 2}=\Gamma^1\times\Gamma^2$ and the volume of this accessible volume of phase space is given by $|\Gamma^{1\cup 2}|=|\Gamma^1||\Gamma^2|$. From this, it's easy to see that the entropy of the composite system is given by
\begin{eqnarray*}
	S &=& k_B\ln|\Gamma^{1\cup 2}| = k_B\ln(|\Gamma^1||\Gamma^2|)\\
		&=& k_B\ln|\Gamma^1| + k_B\ln|\Gamma^2| = S^1 + S^2,
\end{eqnarray*}
which is fortunate, since entropy is an extensive variable and we therefore expect it to be additive.
In this example the two sub-systems were completely isolated from each other; the dynamics of one system had no influence on the dynamics of the other. This condition of dynamic independence corresponds to the independence of the observables that pertain to these sub-systems.

Now, let's look at how the entropy of a composite system changes if we allow for the exchange of energy between the two sub-systems (a closed system). This has the effect of increasing the accessible volume of the phase space, since exchange of energy means that there are more possible configurations for the overall system.

Without energy exchange, the volume of the accessible phase space to the total system is given by
$$
	|\Gamma_0| = |\Gamma^1||\Gamma^2|.
$$
Once we allow for the exchange of energy, this becomes
$$
	|\Gamma| = \sum_{E^1}|\Gamma^1(E^1)||\Gamma^2(E_\text{tot}-E^1)|.
$$
That is, we must now consider all possible configurations where the total energy of the composite system is partitioned between the two sub-systems. This volume is bounded below by $|\Gamma_0|$ since the expression for $|\Gamma_0|$ is just one of the terms in the sum. At first glance, it may look like this increase in the volume of the accessible phase space is enormous (we have added many more possible configurations), however, the volume of the accessible phase space for the composite system corresponds almost entirely to the states where $E^1=E^1_\text{eqm}$ and $E^2=E^2_\text{eqm}$. As a consequence, for large enough $N$, the difference between $\Gamma$ and $|\Gamma^1(E^1_\text{eqm})||\Gamma^2(E^2_\text{eqm})|$ remains small. It's not too hard to show (see section 3.7 of \emph{Statistical Mechanics in a Nutshell}) that the contribution to the accessible phase space volume due to the exchange of energy between the two systems is of order $\sqrt{N}$ compared with the total system size $N$.

\subsection{The canonical ensemble}

Rather than considering a perfectly isolated system, for which energy is conserved, a more realistic experimental situation may be to consider a system S in thermal contact with some much larger reservoir R. This has the effect of holding the total system at constant temperature. In such a situation we want to be able to calculate the average value $\langle A\rangle$ of an observable $A$ for the system S; we are not interested in the state of the reservoir R, except to the extent that it helps us determine the state of S.

If, for the composite system S$\cup$R, we write $x_S,x_R$ for points in the phase space of the microscopic system, then the average value of the observable $A$ is given by
$$
	\langle A\rangle = \frac{1}{|\Gamma|}\int_{\Gamma}dx_Sdx_RA(x_s),
$$
where $\Gamma$ is the region of the phase space for the composite system, when it has total internal energy of $E$.

In order to make explicit the parts of the total phase space that is accessible to the composite system, wewrite the above expression as the Cartesian product of the phase space for the system and the reservoir, i.e.
$$
	\langle A\rangle =\frac{1}{|\Gamma|}\int dx_SA(x_S)\times\int dx_R\delta(H^R(x_R)-(E-H^S(x_S))).
$$
The delta function in the last term is zero, except when $x_S$ and $x_R$ in the two sub-systems take values such that $H^S+H^R=E$. That is the delta function defines the accessible phase space volume when the two sub-systems can exchange internal energy between them, but the total internal energy of the composite system is conserved.

Recalling the fundamental postulate of statistical mechanics ($S=k_B\ln|\Gamma|$), we rewrite the last expression to replace the phase space volume with the corresponding expression for entropy:
$$
	\int dx_R\delta(H^R(x_R)-(E-H^S(x_S))) \simeq \exp\left(\frac{1}{k_B}S^R(E-H^S)\right).
$$

Since $H^S$ is much less than $E$ it makes sense to now expand the entropy term on the right-hand side in a Taylor series about $E$:
$$
	\exp\left(\frac{1}{k_B}S^R(E-H^S)\right)\simeq \exp\left(\frac{1}{k_B}S^R(E)\right)\exp\left(\frac{-1}{k_B}\frac{\partial S^R}{\partial E}|_E H^S(x_S)\right)\ldots
$$

Earlier in the course, we associated $\frac{\partial S}{\partial E}$ with $\frac{1}{T}$; for a canonical ensemble, $T$ is constant (that's the point of the reservoir) so we write the expected value of the observable as
$$
	\langle A\rangle \simeq \frac{1}{|\Gamma|}\int dx_sA(x_S)\times\exp\left(\frac{1}{k_B}s^R(E)\right)\exp\left(\frac{-H^S(x_S)}{k_BT}\right).
$$

It remains to make the normalisation precise. When we do this, the factors of $S\exp\left(\frac{1}{k_B}S^R(E)\right)$ cancel from the integral and its normalisation and we get
\begin{equation}
	\langle A\rangle = \frac{1}{Z}\int dx_SA(x_S)\exp\left(\frac{-H^S(x_S)}{k_BT}\right),
	\label{eq:z1}
\end{equation}
where $Z$ is the known as the \emph{partition function} and is given by
\begin{equation}
	Z = \int dx_S\exp\left(\frac{-H^S(x_S)}{k_BT}\right).
	\label{eq:z2}
\end{equation}

One way to think about equations \ref{eq:z1} and \ref{eq:z2} is that we no longer need to keep track of which parts of phase space are accessible. Instead we can integrate over the entire phase space and each region is weighted appropriately, according to $\exp\frac{-H}{k_BT}$ --- the so-called \emph{Boltzmann factor}, a probability density in the phase space.

Similar to the case of the micro-canonical ensemble, for the canonical ensemble, the contributions to $A$ are dominated by the part of phase space that corresponds to when the system's internal energy is at the equilibrium value. 

You should read through, and understand, section 3.12 of \emph{Statistical Mechanics in a Nutshell} and sections 4.1--4.3 of \emph{Statistical Mechanics Made Simple} for some extra details and explanation.


\subsection{A generalised ensemble}
We can follow a similar approach to what we did for the canonical ensemble and generalise to other types of ensembles. Let's start by considering a system $S$ for which we want to specify a fixed value for $f_i$, some intensive variable, with $X_i$ as the conjugate extensive variable ($\frac{\partial S}{\partial X_i}=-\frac{f_i}{T}$). We can do this by putting the system in contact with a reservoir $R$ with which it can exchange $X_i$ such that $X_i = X_i^S+X_i^R$. Now we follow a similar procedure to that for the canonical ensemble:
\begin{eqnarray*}
	\langle A \rangle &=& \frac{1}{|\Gamma|}\int_{\Gamma}dx_R dx_S A(x_S)\\
	&=& \frac{1}{|\Gamma|}\int dx_S A(x_S)\int dx_R \delta(X_i^R(x_R)+X_i^S(x_S)-X_i)\\
	&=& \frac{1}{|\Gamma|}\int dx_S A(x_S)\exp\left( \frac{-1}{k}S(X_i-X_i^S(x_S))\right)\\
	&\simeq& \frac{1}{Z}\int dx_S A(x_S)\exp\left( \frac{-1}{k}\frac{\partial S}{\partial X_i}X_i^S(x_S)\right)\\
	&=& \frac{1}{Z}\int dx_S A(x_S)\exp\left( \frac{-1}{k} f_i X_i^S(x_S)\right),
\end{eqnarray*}
where the partition function $Z$ is given by
$$
	Z= \int dx \exp\left( \frac{f_iX_i(x)}{kT} \right).
$$
The integral in the partition function can be evaluated via the saddle point method (see the notes about this on Canvas if you want to know more) to get
\begin{equation*}
	Z \simeq \exp\left(\frac{T^S(X_i^*)+f_iX_i^*}{k_BT}\right),
\end{equation*}
where $X_i^*$ is the value of $X_i$ which maximises the value of the exponential.

(See SMiaN, section 3.13 for more on this.)

\subsection{Calculating quantities from a (generalised) ensemble}

If we want to compute the expected value of some extensive variable $X_i$ in our generalised ensemble we have
\[
	\langle X_i\rangle = \frac{1}{Z}\int dx X_i \exp\left(\frac{f_iX_i(x)}{kT}\right) = \frac{\partial \ln Z}{\partial (f_i/kT)}.
\]
Similarly,
\[
	{\partial^2 \ln Z}{\partial (f_i/kT)^2} = \langle X_i^2 \rangle - \langle X_i \rangle^2 = kT\frac{\partial \langle X_i\rangle}{\partial f_i}.
\]

As a specific example of working with the above, let's look at the expected value of the internal energy for a canonical ensemble with constant temperature. First introduce the shorthand $\beta = 1/kT$.
$$
	E = \langle H \rangle = \frac{\int E_S\exp(-\beta H_s)}{\int{\exp(-\beta H_S)}} = \frac{-\partial}{\partial \beta}\ln\int\exp(-\beta H_S).
$$

\subsubsection{Helmholtz free energy}
The \emph{Helmholtz Free Energy} is given by
$$
	F = E - TS
$$
hence
$$
	\mathrm{d}F = \mathrm{d}E - T\mathrm{d}S - S\mathrm{d}T.
$$
This gives $S= -\left(\frac{\partial F}{\partial T}\right)$ and $p = -\left(\frac{\partial F}{\partial V}\right)$. We can use this to write the  internal energy as $E=F+FS=F-T\frac{\partial F}{\partial T}=-T^2\frac{\partial}{\partial T}\left(\frac{F}{T}\right) = \frac{\partial (F/T)}{\partial (1/T)}$. So $E = \frac{\partial}{\partial \beta} F\beta$ but this is also $E = -\frac{\partial}{\partial\beta}\ln(\int\exp(-\beta H))$ so $F= \frac{-\ln(\int\exp(-\beta H))}{\beta} = -kT\ln(Z)$. From this we can calculate the specific heat $C_V = \left(\frac{\partial E}{\partial T} \right)_{N,T} = -T\left(\frac{\partial^2 F}{\partial T^2}\right)_{N,V}$.


\subsubsection{The $P-T$ ensemble}
The \emph{$p-T$ ensemble} is one specific example of a generalised ensemble. In this case the pressure and temperature are fixed, while the internal energy and volume (their conjugate variables) are allowed to fluctuate.
Using the generalised formula (see the sections above), and dropping the subscripts, the $p-T$ ensemble is given by
 \begin{equation*}
	\langle A\rangle = \frac{1}{Z}\int dxA(x)\exp\left(-\frac{E(x)+pV(x)}{k_BT}\right),
\end{equation*}
while the partition function is given by
$$
	\ln Z = -\frac{E-TS+pV}{k_BT}.
$$
The quantity on the top of the fraction is the \emph{Gibbs free energy}. (See section 3.14 of SMiaN)


\subsection{The grand-canonical ensemble}
Rather than holding $N$, the number of particles, fixed we can allow it to vary and instead specify $\mu$ the chemical potential. The corresponding ensemble for this case is the so-called \emph{grand-canonical ensemble}. Following a similar process as for the canonical emsemble we proceed by noting $N^S+N^R = N$ and $N^S<<N^R$. 

The expected value of an observable $A$ is given by
$$
	\langle A\rangle = \frac{1}{Z}\sum_{N=1}^{\infty}\int\mathrm{d}xA(x)\exp\left(-\frac{H_N(x)-\mu N}{k T}\right)
$$
and the corresponding partition function is given by
$$
	Z=\exp\left(-\frac{E-TS-\mu N}{k_B T}\right).
$$

(See section 3.15 of SMiaN and section 4.4 of SMMS for more discussion of the grand canonical ensemble.)

\subsection{Information theory and the Gibbs Formula for entropy}
To finish off this section, we'll look at one last application of entropy; not because it is particularly useful to physical systems in statistical mechanics, but because it gives a result that forms the basis of information theory.

We'll return to a considering a generalised ensemble, like the one we looked at a couple of sections earlier. However, in this case we'll assume that the phase space is discretized and that the index $n$ runs over all of the microstates of the system. If we consider an intensive variable $f$ and its corresponding extensive variable $X$ then the expression for the expected value of any observable $A$ is
$$
	\langle A \rangle = \frac{1}{Z}\sum_n A_n\exp\left(\frac{fX_n}{k_BT}\right),
$$
while the partition function $Z$ is given by
$$
	Z =\sum_n\exp\left(\frac{fX_n}{k_BT}\right).
$$

The partition function is related to the thermodynamic potentials (see the previous section on generalised ensembles and SMiaN sections 3.12 and 3.13) via
\begin{equation}
	\ln Z = \frac{TS+f\langle X\rangle}{kT}.
	\label{eq:gibbZ}
\end{equation}

Now, for any individual microstate $n$ the probability is therefore given by
$$
	p_n = \frac{1}{Z}\exp\left(\frac{fX_n}{k_BT}\right).
$$
Taking the log of both sides of this expression gives
$$
	\ln p_n = \frac{fX_n}{k_BT} - ln Z,
$$
which after substituting \ref{eq:gibbZ} for $\ln Z$ gives
$$
	\ln p_n = \frac{1}{k_BT}(fX_n -TS -f\langle X\rangle).
$$

Now we can calculate the expected value of both side of the above equation
$$
	\langle \ln p_n \rangle = \sum_n p_n\ln p_n = \frac{1}{k_BT}(f\langle X\rangle -TS -\langle X \rangle) = \frac{-S}{kB}.
$$


After rearranging this for $S$ (and making explicit the sum for the expected value of $p_n$) we arrive at the \emph{Gibbs formula for entropy}:
$$
	S = -k_B\sum_np_n\ln p_n.
$$

Although this is elegant, it's generally useless in the context of physical systems since the number of microstates it would be necessary to sum over is far to large to be computationally practical and, in any case, we generally don't know the probability distribution $p_n$ for each of the microstates. The value of this expression lies in its application to other systems, particularly information theory, where it can be used to quantify amount of information in, for example a digital signal, in which case the $p_n$ represent the probability of receiving the $n$-th possible value in the list of signals. We'll also see, later in the course that the Gibbs entropy plays a role in the statistical mechanics of complex networks.

\subsection{Recommended reading}
Most of the notes in this section follow closely the second half of chapter 3 in \emph{Statistical Mechanics in a Nutshell}; specifically, section 3.6--3.18. Chapter four of \emph{Statistical Mechanics made Simple} covers the same material in sections 4.0--4.4. It is gives some intuitive and succinct explanations, but I find it to be of more use \emph{after} you've already looked at SMiaN. Also useful, and with a slightly different presentation (perhaps with more traditional notation), is \emph{Entropy, Order Parameters, and Complexity}. Here the content is spread around a bit over chapters three through six. Much of the useful content, including some relevant to early sections of these notes, is in sections 3.1, 3.5, 6.1, 6.2, 6.3, and 5.3.
