The aim of statistical mechanics is to describe the thermodynamic properties of complex systems, composed of a large number of particles. The evolution times for these systems are microscopic, and it’s more practical to measure changes at the macroscopic level, assuming it provides an average change of microstates.\\
Liouvilles Theorem states that the volume V enclosed by a closed surface in phase space is constant as the surface evolves (or moves through phase space). \\
Phase space is the set of all possible states of the system, with each state being a point in phase space. The initial conditions define a surface (or volume, depending on the dimension) in phase space, and as the system evolves the points move through phase space (and so does our surface). \\
Say we have a point z(q,p) moving through phase space, it’s velocity is given by $\dot z (\dot q,\dot p)=(\frac{dH}{dp}, \frac{-dH}{dq})$. If we have many points, then we can find the change in volume using divergence theorem $\int\nabla\cdot v\d V$, where $\nabla\cdot v=\frac{\partial}{\partial\ q}\ (\dot q)+\frac{\partial}{\partial\ p}\ (\dot p)=\frac{\partial}{\partial\ q}\ (\frac{dH}{dp})+\frac{\partial}{\partial\ p}\ (\frac{-dH}{dq})=0$\\ 

We can also talk about the density of states being constant $\frac{d\rho}{dt} = 0$ which means that after some time t passes the system has the same probability for each microstate, which means we can look at the macrostate to determine what happens to microstates. 
