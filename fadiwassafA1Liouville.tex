\subsection{Fadi Wassaf}

\noindent Let $\Gamma$ be the set of points in your phase space that satisfy the equations of motion for your system. Liouville's theorem states that for an initial $\Gamma^1$ and some $\Gamma^2$ at a later point in time, then $|\Gamma^1| = |\Gamma^2|$. In other words, phase space volumes are conserved through time evolution for a Hamiltonian system. An alternative description may be that $\rho$, the local density (or probability density) of microstates, will stay constant along the trajectories through phase space.\\

\noindent If this wasn't true, then we might not have the ability to work with microstates the way we do now. For example, if the volume (which corresponds to the number of microstates) increased with time evolution, then we'd suddenly new microstates magically coming into existence over time. Given that we have defined entropy as $S=k\ln |\Gamma|$, this would suggest that the entropy of the system would keep increasing and not stay to an equilibrium point.\\

\noindent Liouville's theorem is useful when talking about systems in equilibrium. Given the restriction that\\ $\partial \rho / \partial t = 0$, we know that all ensemble averages will not depend on time. An ensemble where this is the case is stationary. We can then represent a system at equilibrium by a stationary ensemble since the probability distribution is independent of time.