\documentclass{article}
\usepackage[utf8]{inputenc}
\usepackage{amsmath}
\usepackage{mathtools}
\usepackage{subfigure}
\usepackage{graphicx}
\usepackage{physics}
\graphicspath{ {./images/} }
\usepackage{float}

\title{GrayHunterA1Liouville}
\author{ghun245 }
\date{March 2020}

\begin{document}

\maketitle

Liouville's Equation may be most succinctly stated in terms of a Poisson Bracket as:
$$
\pdv{\rho}{t} = -\{\rho,H\}
$$
While deceptively simple, we may consider that the general form of Hamilton's equations in similar notation can be succinctly stated as: for some function $f(p,q,t)$ $\dv{f}{t} = \{f,H \} + \pdv{f}{t}$ This shows by construction that the total change in phase space density over time is zero for Hamiltonian systems. If we consider this phase space density to be representative of the system, we essentially have the blindingly obvious statement that particles cannot vacate a region of space without leaving another region full. \\
Consider a pulse of light and forget for the time being that light is quantised and wavey and all that, we may describe each photon as having some $p_i, q_i$ for each dimension. Constructing the Hamiltonian for such a system would be a feat for the ages, but we can, using Liouville's Theorem, assert that if in general the particles are to be found on the moon after some seconds, that they cannot infact be on Earth anymore.
\end{document}
